\documentclass[12pt,a4paper]{report}

% Packages
\usepackage[utf8]{inputenc}
\usepackage{graphicx}
\usepackage{geometry}
\usepackage{hyperref}
\usepackage{listings}
\usepackage{color}
\usepackage{xcolor}
\usepackage{float}
\usepackage{booktabs}
\usepackage{titlesec}

% Geometry
\geometry{
 a4paper,
 total={170mm,257mm},
 left=20mm,
 top=20mm,
}

% Code Listing Style
\definecolor{codegreen}{rgb}{0,0.6,0}
\definecolor{codegray}{rgb}{0.5,0.5,0.5}
\definecolor{codepurple}{rgb}{0.58,0,0.82}
\definecolor{backcolour}{rgb}{0.95,0.95,0.92}

\lstdefinestyle{mystyle}{
    backgroundcolor=\color{backcolour},   
    commentstyle=\color{codegreen},
    keywordstyle=\color{magenta},
    numberstyle=\tiny\color{codegray},
    stringstyle=\color{codepurple},
    basicstyle=\ttfamily\footnotesize,
    breakatwhitespace=false,         
    breaklines=true,                 
    captionpos=b,                    
    keepspaces=true,                 
    numbers=left,                    
    numbersep=5pt,                  
    showspaces=false,                
    showstringspaces=false,
    showtabs=false,                  
    tabsize=2
}

\lstset{style=mystyle}

% Title Format
\titleformat{\chapter}[display]
  {\normalfont\bfseries}{}{0pt}{\Huge}

% Document Info
\title{\textbf{Social Media Backend Application Documentation}}
\author{Harsh Upadhyay}
\date{\today}

\begin{document}

\maketitle

\tableofcontents

\chapter{Introduction}
\section{Overview}
This document details the architecture, design, and implementation of a robust backend system for a social media application. The application is built using a modern technology stack centered around Node.js and MongoDB, providing a scalable and efficient platform for users to share content and interact.

\section{Purpose}
The primary purpose of this application is to facilitate user interaction through post sharing. Key features include:
\begin{itemize}
    \item Secure user authentication (Registration \& Login).
    \item Image uploading and hosting.
    \item Post creation with captions.
    \item Browsing user feeds.
    \item Follow/Unfollow mechanism (Data model provided).
\end{itemize}

\chapter{System Architecture}
\section{Architectural Pattern}
The application follows the **Model-View-Controller (MVC)** architectural pattern, which separates the application into three main logical components:
\begin{itemize}
    \item \textbf{Model:} Represents the data structure and business logic. It interacts directly with the database. In our case, these are the Mongoose schemas defining Users, Posts, and Follows.
    \item \textbf{View:} Represents the visual interface. Since this is a backend API, the "View" is represented by the JSON responses sent to the client (React, Mobile App, etc.).
    \item \textbf{Controller:} Acts as an intermediary between Model and View. It processes incoming requests, interacts with the Model to fetch or update data, and returns the appropriate response.
\end{itemize}

\section{Directory Structure}
The project is organized as follows:
\begin{verbatim}
/src
  |-- /config       # Configuration files (Database connection)
  |-- /controllers  # Request handlers (Business Logic)
  |-- /middlewares  # Express Middleware (Authentication)
  |-- /models       # Mongoose Schemas (Data Models)
  |-- /routes       # API Route definitions
  |-- app.js        # Express App setup
\end{verbatim}

\chapter{Technologies and Tools}
The following technologies were selected for this project:

\section{Node.js}
\textbf{What it is:} A JavaScript runtime built on Chrome's V8 JavaScript engine.
\textbf{Why it is used:} It allows us to use JavaScript on the server-side, enabling a unified language for both frontend and backend. Its event-driven, non-blocking I/O model makes it lightweight and efficient for real-time applications.

\section{Express.js}
\textbf{What it is:} A minimal and flexible Node.js web application framework.
\textbf{Why it is used:} It simplifies the process of building robust APIs by providing a thin layer of fundamental web application features, such as routing and middleware support, without obscuring Node.js features.

\section{MongoDB \& Mongoose}
\textbf{What it is:} MongoDB is a NoSQL database program, using JSON-like documents with optional schemas. Mongoose is an Object Data Modeling (ODM) library for MongoDB and Node.js.
\textbf{Why it is used:} MongoDB's document model maps naturally to objects in application code, making it easy to work with. Mongoose provides a straightforward, schema-based solution to model application data, including built-in type casting, validation, and query building.

\section{JSON Web Token (JWT)}
\textbf{What it is:} An open standard (RFC 7519) that defines a compact and self-contained way for securely transmitting information between parties as a JSON object.
\textbf{Why it is used:} Used for stateless authentication. Once a user logs in, the server generates a token signed with a secret key. The client sends this token with subsequent requests, allowing the server to verify the user's identity without querying a session store.

\section{Bcrypt.js}
\textbf{What it is:} A library to help you hash passwords.
\textbf{Why it is used:} Security. Storing passwords in plain text is a critical vulnerability. Bcrypt hashes passwords with a salt, making them resistant to rainbow table attacks.

\section{ImageKit}
\textbf{What it is:} A cloud-based image management solution.
\textbf{Why it is used:} To store and serve user-uploaded images efficiently. It handles image optimization and delivery via CDN, offloading heavy media handling from our main server.

\chapter{Database Design}
The database consists of three primary collections: Users, Posts, and Follows.

\section{User Model}
Stores user profile information and authentication credentials.
\begin{table}[H]
\centering
\begin{tabular}{|l|l|l|}
\hline
\textbf{Field} & \textbf{Type} & \textbf{Description} \\ \hline
username & String & Unique identifier for the user. \\ \hline
email & String & Unique email address. \\ \hline
password & String & Hashed password string. \\ \hline
bio & String & Short user biography. \\ \hline
profileImage & String & URL to the profile picture. \\ \hline
\end{tabular}
\caption{User Schema}
\end{table}

\section{Post Model}
Stores content created by users.
\begin{table}[H]
\centering
\begin{tabular}{|l|l|l|}
\hline
\textbf{Field} & \textbf{Type} & \textbf{Description} \\ \hline
caption & String & Text accompanying the image. \\ \hline
imgUrl & String & URL of the uploaded image (from ImageKit). \\ \hline
user & ObjectId & Reference to the User who created the post. \\ \hline
\end{tabular}
\caption{Post Schema}
\end{table}

\section{Follow Model}
Manages the many-to-many relationship between users (Followers/Following).
\begin{table}[H]
\centering
\begin{tabular}{|l|l|l|}
\hline
\textbf{Field} & \textbf{Type} & \textbf{Description} \\ \hline
follower & ObjectId & Reference to the User who is following. \\ \hline
followee & ObjectId & Reference to the User being followed. \\ \hline
timestamps & Date & Creation/Update time. \\ \hline
\end{tabular}
\caption{Follow Schema}
\end{table}

\chapter{Implementation Details}
This chapter explores key components of the implementation.

\section{Authentication}
Authentication is handled in `auth.controller.js`.
\subsection{Registration}
The registration process involves:
1. Validating input.
2. Checking if the user already exists.
3. Hashing the password.
4. Creating the user record.
5. Generating a JWT token.

\begin{lstlisting}[language=JavaScript, caption=User Registration Implementation]
async function registerController(req, res) {
    const { email, username, password, bio, profileImage } = req.body
    
    // Check if user exists
    const isUserAlreadyExists = await userModel.findOne({
        $or: [{ username }, { email }]
    })
    
    if (isUserAlreadyExists) {
        return res.status(409).json({ message: "User already exists" })
    }
    
    // Hash password
    const hash = await bcrypt.hash(password, 10)
    
    // Create User
    const user = await userModel.create({
        username, email, bio, profileImage, password: hash
    })
    
    // Generate Token
    const token = jwt.sign({ id: user._id }, process.env.JWT_SECRET, { expiresIn: "1d" })
    
    res.cookie("token", token)
    res.status(201).json({ message: "User Registered successfully", user })
}
\end{lstlisting}

\section{Middleware}
Protected routes use the `identifyUser` middleware to verify the JWT token.
\begin{lstlisting}[language=JavaScript, caption=Auth Middleware]
async function identifyUser(req, res, next) {
    const token = req.cookies.token
    if (!token) return res.status(401).json({ message: "Token not provided" })

    try {
        const decoded = jwt.verify(token, process.env.JWT_SECRET)
        req.user = decoded // Attach user info to request
        next()
    } catch (err) {
        return res.status(401).json({ message: "user not authorized" })
    }
}
\end{lstlisting}

\section{Post Management}
Posts are created with image uploads. We use `multer` to handle the file in memory and then upload it to `ImageKit`.
\begin{lstlisting}[language=JavaScript, caption=Creating a Post]
async function createPostController(req, res) {
    // Upload implementation details...
    const file = await imagekit.files.upload({
        file: await toFile(Buffer.from(req.file.buffer), 'file'), // handled by multer
        fileName: "PostImage",
    })

    const post = await postModel.create({
        caption: req.body.caption,
        imgUrl: file.url,
        user: req.user.id
    })

    res.status(201).json({ message: "Post created", post })
}
\end{lstlisting}

\chapter{API Endpoints}
\begin{table}[H]
\centering
\begin{tabular}{|l|l|l|l|}
\hline
\textbf{Method} & \textbf{Endpoint} & \textbf{Protected} & \textbf{Description} \\ \hline
POST & /api/auth/register & No & Register a new user. \\ \hline
POST & /api/auth/login & No & Login existing user. \\ \hline
POST & /api/posts & Yes & Create a new post (w/ image). \\ \hline
GET & /api/posts & Yes & Get posts for current user. \\ \hline
GET & /api/posts/details/:id & Yes & Get details of a specific post. \\ \hline
\end{tabular}
\caption{API Endpoints}
\end{table}

\chapter{Conclusion}
This documentation has outlined the structure and implementation of the Social Media Backend Application. By leveraging Node.js, Express, and MongoDB, we have created a scalable foundation. The implementation of JWT authentication ensures security, while ImageKit integration manages media efficiently. The MVC architecture ensures code maintainability and separation of concerns.

\end{document}
